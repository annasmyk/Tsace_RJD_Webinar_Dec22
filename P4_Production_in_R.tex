\PassOptionsToPackage{unicode=true}{hyperref} % options for packages loaded elsewhere
\PassOptionsToPackage{hyphens}{url}
%
\documentclass[10pt,xcolor=table,color={dvipsnames,usenames},ignorenonframetext,usepdftitle=false,french]{beamer}
\setbeamertemplate{caption}[numbered]
\setbeamertemplate{caption label separator}{: }
\setbeamercolor{caption name}{fg=normal text.fg}
\beamertemplatenavigationsymbolsempty
\usepackage{caption}
\captionsetup{skip=0pt,belowskip=0pt}
%\setlength\abovecaptionskip{-15pt}
\usepackage{lmodern}
\usepackage{amssymb,amsmath,mathtools,multirow}
\usepackage{float,hhline}
\usepackage{tikz}
\usepackage{mathtools}
\usepackage{ifxetex,ifluatex}
\usepackage{fixltx2e} % provides \textsubscript
\ifnum 0\ifxetex 1\fi\ifluatex 1\fi=0 % if pdftex
  \usepackage[T1]{fontenc}
  \usepackage[utf8]{inputenc}
  \usepackage{textcomp} % provides euro and other symbols
\else % if luatex or xelatex
  \usepackage{unicode-math}
  \defaultfontfeatures{Ligatures=TeX,Scale=MatchLowercase}
\fi
\usetheme[coding=utf8,language=english,
,titlepagelogo=img/SACElogo
]{TorinoTh}
% use upquote if available, for straight quotes in verbatim environments
\IfFileExists{upquote.sty}{\usepackage{upquote}}{}
% use microtype if available
\IfFileExists{microtype.sty}{%
\usepackage[]{microtype}
\UseMicrotypeSet[protrusion]{basicmath} % disable protrusion for tt fonts
}{}
\IfFileExists{parskip.sty}{%
\usepackage{parskip}
}{% else
\setlength{\parindent}{0pt}
\setlength{\parskip}{6pt plus 2pt minus 1pt}
}
\usepackage{hyperref}
\hypersetup{
            pdfauthor={Anna Smyk and Tanguy Barthelemy},
            pdfborder={0 0 0},
            breaklinks=true}
\urlstyle{same}  % don't use monospace font for urls
\newif\ifbibliography
\newlength{\cslhangindent}
\setlength{\cslhangindent}{1.5em}
\newlength{\csllabelwidth}
\setlength{\csllabelwidth}{3em}
\newenvironment{CSLReferences}[2] % #1 hanging-ident, #2 entry spacing
 {% don't indent paragraphs
  \setlength{\parindent}{0pt}
  % turn on hanging indent if param 1 is 1
  \ifodd #1 \everypar{\setlength{\hangindent}{\cslhangindent}}\ignorespaces\fi
  % set entry spacing
  \ifnum #2 > 0
  \setlength{\parskip}{#2\baselineskip}
  \fi
 }%
 {}
\usepackage{color}
\usepackage{fancyvrb}
\newcommand{\VerbBar}{|}
\newcommand{\VERB}{\Verb[commandchars=\\\{\}]}
\DefineVerbatimEnvironment{Highlighting}{Verbatim}{commandchars=\\\{\}}
% Add ',fontsize=\small' for more characters per line
\usepackage{framed}
\definecolor{shadecolor}{RGB}{248,248,248}
\newenvironment{Shaded}{\begin{snugshade}}{\end{snugshade}}
\newcommand{\AlertTok}[1]{\textcolor[rgb]{0.94,0.16,0.16}{#1}}
\newcommand{\AnnotationTok}[1]{\textcolor[rgb]{0.56,0.35,0.01}{\textbf{\textit{#1}}}}
\newcommand{\AttributeTok}[1]{\textcolor[rgb]{0.77,0.63,0.00}{#1}}
\newcommand{\BaseNTok}[1]{\textcolor[rgb]{0.00,0.00,0.81}{#1}}
\newcommand{\BuiltInTok}[1]{#1}
\newcommand{\CharTok}[1]{\textcolor[rgb]{0.31,0.60,0.02}{#1}}
\newcommand{\CommentTok}[1]{\textcolor[rgb]{0.56,0.35,0.01}{\textit{#1}}}
\newcommand{\CommentVarTok}[1]{\textcolor[rgb]{0.56,0.35,0.01}{\textbf{\textit{#1}}}}
\newcommand{\ConstantTok}[1]{\textcolor[rgb]{0.00,0.00,0.00}{#1}}
\newcommand{\ControlFlowTok}[1]{\textcolor[rgb]{0.13,0.29,0.53}{\textbf{#1}}}
\newcommand{\DataTypeTok}[1]{\textcolor[rgb]{0.13,0.29,0.53}{#1}}
\newcommand{\DecValTok}[1]{\textcolor[rgb]{0.00,0.00,0.81}{#1}}
\newcommand{\DocumentationTok}[1]{\textcolor[rgb]{0.56,0.35,0.01}{\textbf{\textit{#1}}}}
\newcommand{\ErrorTok}[1]{\textcolor[rgb]{0.64,0.00,0.00}{\textbf{#1}}}
\newcommand{\ExtensionTok}[1]{#1}
\newcommand{\FloatTok}[1]{\textcolor[rgb]{0.00,0.00,0.81}{#1}}
\newcommand{\FunctionTok}[1]{\textcolor[rgb]{0.00,0.00,0.00}{#1}}
\newcommand{\ImportTok}[1]{#1}
\newcommand{\InformationTok}[1]{\textcolor[rgb]{0.56,0.35,0.01}{\textbf{\textit{#1}}}}
\newcommand{\KeywordTok}[1]{\textcolor[rgb]{0.13,0.29,0.53}{\textbf{#1}}}
\newcommand{\NormalTok}[1]{#1}
\newcommand{\OperatorTok}[1]{\textcolor[rgb]{0.81,0.36,0.00}{\textbf{#1}}}
\newcommand{\OtherTok}[1]{\textcolor[rgb]{0.56,0.35,0.01}{#1}}
\newcommand{\PreprocessorTok}[1]{\textcolor[rgb]{0.56,0.35,0.01}{\textit{#1}}}
\newcommand{\RegionMarkerTok}[1]{#1}
\newcommand{\SpecialCharTok}[1]{\textcolor[rgb]{0.00,0.00,0.00}{#1}}
\newcommand{\SpecialStringTok}[1]{\textcolor[rgb]{0.31,0.60,0.02}{#1}}
\newcommand{\StringTok}[1]{\textcolor[rgb]{0.31,0.60,0.02}{#1}}
\newcommand{\VariableTok}[1]{\textcolor[rgb]{0.00,0.00,0.00}{#1}}
\newcommand{\VerbatimStringTok}[1]{\textcolor[rgb]{0.31,0.60,0.02}{#1}}
\newcommand{\WarningTok}[1]{\textcolor[rgb]{0.56,0.35,0.01}{\textbf{\textit{#1}}}}
% Prevent slide breaks in the middle of a paragraph:
\widowpenalties 1 10000
\raggedbottom
\AtBeginPart{
  \let\insertpartnumber\relax
  \let\partname\relax
  \frame{\partpage}
}
\AtBeginSection{
  \ifbibliography
  \else
    \begin{frame}[noframenumbering]{Contents}
    \tableofcontents[currentsection, hideothersubsections]
    \end{frame}
  \fi
}
\setlength{\emergencystretch}{3em}  % prevent overfull lines
\providecommand{\tightlist}{%
  %\setlength{\itemsep}{0pt}
  \setlength{\parskip}{0pt}
  }
\setcounter{secnumdepth}{0}

% set default figure placement to htbp
\makeatletter
\def\fps@figure{htbp}
\makeatother

\usepackage{dsfont}
\usepackage{stmaryrd}
\usepackage[normalem]{ulem}
\usepackage{fontawesome5}
\usepackage{tikz,pgfplots}
\pgfplotsset{compat=1.17}
\pgfplotsset{samples=100}
\usepackage{animate}
 \usepackage{booktabs}

\usepackage{colortbl}

\DeclareMathOperator{\Cov}{Cov}
\newcommand{\cov}[2]{\Cov\left( #1\,,\,#2 \right)}

\DeclareMathOperator{\e}{e}
\renewcommand{\P}{\mathds{P}} %Apparement \P existe déjà ?
\newcommand\N{\mathds{N}}
\newcommand\R{\mathds{R}}


\newcommand\1{\mathds{1}}
\newcommand{\E}[2][]{{\mathds{E}}_{#1}
  \def\temp{#2}\ifx\temp\empty
  \else
    \left[#2\right]
  \fi
}
\newcommand{\V}[2][]{{\mathds{V}}_{#1}
  \def\temp{#2}\ifx\temp\empty
  \else
    \left[#2\right]
  \fi
}
\newcommand\ud{\,\mathrm{d}}


% blocks
\usepackage{environ}
\usepackage[tikz]{bclogo}

\tikzstyle{titlestyle} =[draw=black!80,fill=black!20, text=black,
 right=10pt, rounded corners]
\mdfdefinestyle{symmaryboxstyle}{
	linecolor=black!80, backgroundcolor = black!5,
	skipabove=\baselineskip, innertopmargin=\baselineskip,
	innerbottommargin=\baselineskip,
	userdefinedwidth=\textwidth,
	middlelinewidth=1.2pt, roundcorner=5pt,
	skipabove={\dimexpr0.5\baselineskip+\topskip\relax},
	frametitleaboveskip=\dimexpr-\ht\strutbox\relax,
	innerlinewidth=0pt,
}
\NewEnviron{summary}{%
\begin{mdframed}[style=symmaryboxstyle]
\vspace{-0.5em}
\BODY
\end{mdframed}
}
\makeatletter
% Open `\noalign` and check for overlay specification:
\def\rowcolor{\noalign{\ifnum0=`}\fi\bmr@rowcolor}
\newcommand<>{\bmr@rowcolor}{%
    \alt#1%
        {\global\let\CT@do@color\CT@@do@color\@ifnextchar[\CT@rowa\CT@rowb}% Rest of original `\rowcolor`
        {\ifnum0=`{\fi}\@gooble@rowcolor}% End `\noalign` and gobble all arguments of `\rowcolor`.
}
% Gobble all normal arguments of `\rowcolor`:
\newcommand{\@gooble@rowcolor}[2][]{\@gooble@rowcolor@}
\newcommand{\@gooble@rowcolor@}[1][]{\@gooble@rowcolor@@}
\newcommand{\@gooble@rowcolor@@}[1][]{\ignorespaces}

\newcommand{\rowc}[1]{\only<#1>{\\\rowcolor{processblue!40}}}
%\newcommand{\rowc}[1]{{\rowcolor<#1>{processblue!30}}
\newcommand{\cellc}[1]{\only<#1>{\cellcolor{processblue!40}}}
\newcommand{\supsp}[1]{\visible<#1>{\\}}

\title{Using JDemetra+ in R: from version 2 to version 3\\
Presentation 4: SA production and qulaity report in R}
\ateneo{TSACE Webinar, Wednesday December 14th 2022}
\author{Anna Smyk and Tanguy Barthelemy}
\date{}


\setrellabel{}

\setcandidatelabel{}

\rel{}
\division{With the collaboration of Alain Quartier-la-tente\\}

\departement{}
\makeatletter
\let\@@magyar@captionfix\relax
\makeatother


\begin{document}
\begin{frame}[plain,noframenumbering]
\titlepage
\end{frame}

\hypertarget{introduction}{%
\section{Introduction}\label{introduction}}

\begin{frame}{Tackling Production issues}
\protect\hypertarget{tackling-production-issues}{}
massive data sets
\end{frame}

\hypertarget{quality-report-with-jdcruncher}{%
\section{Quality report with
JDCruncheR}\label{quality-report-with-jdcruncher}}

\begin{frame}[fragile]{Quality report with JDCruncheR (1/3)}
\protect\hypertarget{quality-report-with-jdcruncher-13}{}
JDemetra+ Cruncher (executable module) allows to - update a JDemetra+
workspace (refresh policy) - export the results (series and
diagnostics), without having to open the graphical interface and operate
manually.

It can be launched in R with rjwsacruncher or JDCrunhcerR packages.

The JDCruncheR package also:

\begin{itemize}
\tightlist
\item
  computes a score (based on ``Good'', ``Bad'' modalities og selected
  diagnostics)
\item
  creates a quality report from the diagnostics produced by JDemetra+
\end{itemize}

The three main functions of the package are:

\begin{itemize}
\item
  \texttt{extract\_QR} to extract the quality report from the csv file
  (demetra\_m.csv) that contains all JD+ diagnostics;
\item
  \texttt{compute\_score} to compute a weighted score based on the
  diagnostics
\item
  \texttt{export\_xlsx} to export the quality report.
\end{itemize}
\end{frame}

\begin{frame}[fragile]{Quality report with JDCruncheR (2/3): example}
\protect\hypertarget{quality-report-with-jdcruncher-23-example}{}
\begin{Shaded}
\begin{Highlighting}[]
\CommentTok{\# choose the demetra\_m.csv file generated by the cruncher}
\NormalTok{QR }\OtherTok{\textless{}{-}} \FunctionTok{extract\_QR}\NormalTok{(}\StringTok{"../Output/SA"}\NormalTok{)}
\NormalTok{QR}

\NormalTok{?compute\_score }\CommentTok{\# to see how the score is calculated (formula)}
\NormalTok{QR }\OtherTok{\textless{}{-}} \FunctionTok{compute\_score}\NormalTok{(QR,}
                    \AttributeTok{n\_contrib\_score =} \DecValTok{3}\NormalTok{)}

\NormalTok{QR}

\NormalTok{QR }\OtherTok{\textless{}{-}} \FunctionTok{sort}\NormalTok{(QR, }\AttributeTok{decreasing =} \ConstantTok{TRUE}\NormalTok{, }\AttributeTok{sort\_variables =} \StringTok{"score"}\NormalTok{)}
\FunctionTok{export\_xlsx}\NormalTok{(QR,}
            \AttributeTok{file\_name =} \StringTok{"U:/quality\_report.xls"}\NormalTok{)}
\end{Highlighting}
\end{Shaded}

When working with several workspaces (or SAPs), quality reports can be
piled up with the function \texttt{rbind()} or by creating a mQR\_matrix
object with the function \texttt{mQR\_matrix()}
\end{frame}

\begin{frame}[fragile]{Quality report with JDCruncher (3/3) : example}
\protect\hypertarget{quality-report-with-jdcruncher-33-example}{}
Missing values can be ignored and conditions can be set for indicators:

\begin{Shaded}
\begin{Highlighting}[]
\CommentTok{\# oos\_mse weight reduced to 1 when the other }
\CommentTok{\# indicators are "Bad" ou "Severe"}
\NormalTok{condition1 }\OtherTok{\textless{}{-}} \FunctionTok{list}\NormalTok{(}\AttributeTok{indicator =} \StringTok{"oos\_mse"}\NormalTok{,}
                   \AttributeTok{conditions =} \FunctionTok{c}\NormalTok{(}\StringTok{"residuals\_independency"}\NormalTok{,}
                                  \StringTok{"residuals\_homoskedasticity"}\NormalTok{,}
                                  \StringTok{"residuals\_normality"}\NormalTok{),}
                   \AttributeTok{conditions\_modalities =} \FunctionTok{c}\NormalTok{(}\StringTok{"Bad"}\NormalTok{,}\StringTok{"Severe"}\NormalTok{))}
\NormalTok{BQ }\OtherTok{\textless{}{-}} \FunctionTok{compute\_score}\NormalTok{(BQ, }\AttributeTok{n\_contrib\_score =} \DecValTok{5}\NormalTok{,}
                    \AttributeTok{conditional\_indicator =} \FunctionTok{list}\NormalTok{(condition1),}
                    \AttributeTok{na.rm =} \ConstantTok{TRUE}\NormalTok{)}
\end{Highlighting}
\end{Shaded}
\end{frame}

\begin{frame}{Example of score composition}
\protect\hypertarget{example-of-score-composition}{}
\begin{table}[htbp]
  \centering
    \begin{tabular}{|m{2cm}|l|>{\centering\arraybackslash}m{2cm}|}
    \hline
    \multicolumn{2}{|c|}{Diagnostics} & Weights (out of 100) \\
    \hline
    \multirow{2}{2cm}{Pre-adjustment} & ARIMA Model Residuals & 30 \\
\cline{2-3}          & Residual Calendar Effects  & 20 \\
    \hline
    \multirow{2}{2cm}{Decomposition} & Residual seasonality & 45 \\
\cline{2-3}          & Decomposition Quality (stats M if X11) & 5 \\
    \hline
    \end{tabular}

\end{table}
\end{frame}

\begin{frame}{Customize the score computation}
\protect\hypertarget{customize-the-score-computation}{}
Practical steps if you want to customize the score computation (see
package documentation in R)

\begin{itemize}
\item
  select your indicators of interest
\item
  adjust ``good'', ``bad''\ldots threshold in JD+ GUI if necessary
\item
  by default good=0, uncertain=1, bad or severe=3
\item
  change this grading system and/or the weights directly in the package
  functions
\item
  rebuild your package
\end{itemize}

\emph{Warning} : here only diagnostics are taken into account, revisions
and numerical effects of potential parameter tuning have to be analysed
with a complementary tool
\end{frame}

\hypertarget{sa-production-in-r}{%
\section{SA production in R}\label{sa-production-in-r}}

\begin{frame}{SA production (fully ?) in R}
\protect\hypertarget{sa-production-fully-in-r}{}
A request wich comes back all the time

\begin{itemize}
\item
  flexibility of data format
\item
  feel of better automatization
\end{itemize}

Here contrast

\begin{itemize}
\item
  ``old fashion set-up'': WS created in GUI, readable with GUI
  refreshable with the cruncher some R might be used for\ldots{}
\item
  ``full R set-up'': no ws structre, time series object
\end{itemize}
\end{frame}

\begin{frame}{Data format and portability}
\protect\hypertarget{data-format-and-portability}{}
\end{frame}

\begin{frame}{Tuning specifications}
\protect\hypertarget{tuning-specifications}{}
(Process set up or annual review)

Massive data set eache series (or goup of series) has specific
(pre-determined) parameters: - pre-specified outliers - calendar
regressors
\end{frame}

\begin{frame}{Estimation and Refreshing data}
\protect\hypertarget{estimation-and-refreshing-data}{}
(Annual or infra anual reviews)

from P2 (everything here or split ?)
\end{frame}

\begin{frame}{Annual review}
\protect\hypertarget{annual-review}{}
comparing old and new sets of params ``current'' params vs automiatic
reestimation (with some user-def params)
\end{frame}

\begin{frame}{Selective editing and Manual fine tuning}
\protect\hypertarget{selective-editing-and-manual-fine-tuning}{}
select series

looking ad diagnostics/ fine tuning params (loop)

with or without GUI

reading data, comparing numerical impact of params
\end{frame}

\hypertarget{conclusion}{%
\section{Conclusion}\label{conclusion}}

\begin{frame}{On production in R}
\protect\hypertarget{on-production-in-r}{}
Assets of WS-GUI-Cruncher set up (with some R help) GUI for manual fine
tuning

Assets of ``Full R set up''
\end{frame}

\hypertarget{overall-conclusion}{%
\subsection{Overall conclusion}\label{overall-conclusion}}

\begin{frame}{Take home message}
\protect\hypertarget{take-home-message}{}
summary

\begin{itemize}
\item
  what is new
\item
  what is missing
\end{itemize}
\end{frame}

\begin{frame}{Possible Contributions}
\protect\hypertarget{possible-contributions}{}
\begin{itemize}
\item
  Testing it and reporting issues
\item
  Developping new tools (other packages, new functions, etc.)
\end{itemize}
\end{frame}

\begin{frame}{Resources}
\protect\hypertarget{resources}{}
\begin{itemize}
\item
  \textbf{Webinar Resources} on GitHub: slides, code, additionnal
  references (Beamers and papers)
  \url{https://github.com/annasmyk/Tsace_RJD_Webinar_Dec22}
\item
  Coming soon: \textbf{JDemetra+ NEW online documentation first release}
  on Thursday december 22nd:
  \url{https://jdemetra-new-documentation.netlify.app/}
\end{itemize}

Restriced scope : SA (incl HF) and Chapter on Tools (GUI, R packages and
plung-ins)

\begin{itemize}
\tightlist
\item
  Blog \textbf{JDemetra+ universe}
  \url{https://jdemetra-universe-blog.netlify.app/}

  \begin{itemize}
  \tightlist
  \item
    can be used for problem/solution/insights sharing (comments
    available if logged into GitHub)
  \item
    guest posts welcome
  \item
    will link ``all'' presentations about JDemetra+ in confs / workshop
    (so if you give a talk about JD+ let us know..)
  \end{itemize}
\end{itemize}
\end{frame}

\begin{frame}[noframenumbering]{Thank you for your attention}
\protect\hypertarget{thank-you-for-your-attention}{}
Packages \faIcon{r-project}{}:

\begin{columns}[T]
\begin{column}{0.4\textwidth}
\href{https://github.com/palatej/rjd3toolkit}{\faGithub{} palatej/rjd3toolkit}

\href{https://github.com/palatej/rjd3modelling}{\faGithub{} palatej/rjd3modelling}

\href{https://github.com/palatej/rjd3sa}{\faGithub{} palatej/rjd3sa}

\href{https://github.com/palatej/rjd3arima}{\faGithub{} palatej/rjd3arima}

\href{https://github.com/palatej/rjd3x13}{\faGithub{} palatej/rjd3x13}

\href{https://github.com/palatej/rjd3tramoseats}{\faGithub{} palatej/rjd3tramoseats}

\href{https://github.com/palatej/rjdemetra3}{\faGithub{} palatej/rjdemetra3}
\end{column}

\begin{column}{0.5\textwidth}
\href{https://github.com/palatej/rjdfilters}{\faGithub{} palatej/rjdfilters}

\href{https://github.com/palatej/rjd3sts}{\faGithub{} palatej/rjd3sts}

\href{https://github.com/palatej/rjd3stl}{\faGithub{} palatej/rjd3stl}

\href{https://github.com/palatej/rjd3highfreq}{\faGithub{} palatej/rjd3highfreq}

\href{https://github.com/palatej/rjd3bench}{\faGithub{} palatej/rjd3bench}

\href{https://github.com/AQLT/ggdemetra3}{\faGithub{} AQLT/ggdemetra3}
\end{column}
\end{columns}
\end{frame}

\end{document}
