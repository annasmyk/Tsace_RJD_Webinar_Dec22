\PassOptionsToPackage{unicode=true}{hyperref} % options for packages loaded elsewhere
\PassOptionsToPackage{hyphens}{url}
%
\documentclass[10pt,xcolor=table,color={dvipsnames,usenames},ignorenonframetext,usepdftitle=false,french]{beamer}
\setbeamertemplate{caption}[numbered]
\setbeamertemplate{caption label separator}{: }
\setbeamercolor{caption name}{fg=normal text.fg}
\beamertemplatenavigationsymbolsempty
\usepackage{caption}
\captionsetup{skip=0pt,belowskip=0pt}
%\setlength\abovecaptionskip{-15pt}
\usepackage{lmodern}
\usepackage{amssymb,amsmath,mathtools,multirow}
\usepackage{float,hhline}
\usepackage{tikz}
\usepackage{mathtools}
\usepackage{ifxetex,ifluatex}
\usepackage{fixltx2e} % provides \textsubscript
\ifnum 0\ifxetex 1\fi\ifluatex 1\fi=0 % if pdftex
  \usepackage[T1]{fontenc}
  \usepackage[utf8]{inputenc}
  \usepackage{textcomp} % provides euro and other symbols
\else % if luatex or xelatex
  \usepackage{unicode-math}
  \defaultfontfeatures{Ligatures=TeX,Scale=MatchLowercase}
\fi
\usetheme[coding=utf8,language=english,
,titlepagelogo=img/SACElogo
]{TorinoTh}
% use upquote if available, for straight quotes in verbatim environments
\IfFileExists{upquote.sty}{\usepackage{upquote}}{}
% use microtype if available
\IfFileExists{microtype.sty}{%
\usepackage[]{microtype}
\UseMicrotypeSet[protrusion]{basicmath} % disable protrusion for tt fonts
}{}
\IfFileExists{parskip.sty}{%
\usepackage{parskip}
}{% else
\setlength{\parindent}{0pt}
\setlength{\parskip}{6pt plus 2pt minus 1pt}
}
\usepackage{hyperref}
\hypersetup{
            pdfauthor={Anna Smyk and Tanguy Barthelemy},
            pdfborder={0 0 0},
            breaklinks=true}
\urlstyle{same}  % don't use monospace font for urls
\newif\ifbibliography
\newlength{\cslhangindent}
\setlength{\cslhangindent}{1.5em}
\newlength{\csllabelwidth}
\setlength{\csllabelwidth}{3em}
\newenvironment{CSLReferences}[2] % #1 hanging-ident, #2 entry spacing
 {% don't indent paragraphs
  \setlength{\parindent}{0pt}
  % turn on hanging indent if param 1 is 1
  \ifodd #1 \everypar{\setlength{\hangindent}{\cslhangindent}}\ignorespaces\fi
  % set entry spacing
  \ifnum #2 > 0
  \setlength{\parskip}{#2\baselineskip}
  \fi
 }%
 {}
\usepackage{color}
\usepackage{fancyvrb}
\newcommand{\VerbBar}{|}
\newcommand{\VERB}{\Verb[commandchars=\\\{\}]}
\DefineVerbatimEnvironment{Highlighting}{Verbatim}{commandchars=\\\{\}}
% Add ',fontsize=\small' for more characters per line
\usepackage{framed}
\definecolor{shadecolor}{RGB}{248,248,248}
\newenvironment{Shaded}{\begin{snugshade}}{\end{snugshade}}
\newcommand{\AlertTok}[1]{\textcolor[rgb]{0.94,0.16,0.16}{#1}}
\newcommand{\AnnotationTok}[1]{\textcolor[rgb]{0.56,0.35,0.01}{\textbf{\textit{#1}}}}
\newcommand{\AttributeTok}[1]{\textcolor[rgb]{0.77,0.63,0.00}{#1}}
\newcommand{\BaseNTok}[1]{\textcolor[rgb]{0.00,0.00,0.81}{#1}}
\newcommand{\BuiltInTok}[1]{#1}
\newcommand{\CharTok}[1]{\textcolor[rgb]{0.31,0.60,0.02}{#1}}
\newcommand{\CommentTok}[1]{\textcolor[rgb]{0.56,0.35,0.01}{\textit{#1}}}
\newcommand{\CommentVarTok}[1]{\textcolor[rgb]{0.56,0.35,0.01}{\textbf{\textit{#1}}}}
\newcommand{\ConstantTok}[1]{\textcolor[rgb]{0.00,0.00,0.00}{#1}}
\newcommand{\ControlFlowTok}[1]{\textcolor[rgb]{0.13,0.29,0.53}{\textbf{#1}}}
\newcommand{\DataTypeTok}[1]{\textcolor[rgb]{0.13,0.29,0.53}{#1}}
\newcommand{\DecValTok}[1]{\textcolor[rgb]{0.00,0.00,0.81}{#1}}
\newcommand{\DocumentationTok}[1]{\textcolor[rgb]{0.56,0.35,0.01}{\textbf{\textit{#1}}}}
\newcommand{\ErrorTok}[1]{\textcolor[rgb]{0.64,0.00,0.00}{\textbf{#1}}}
\newcommand{\ExtensionTok}[1]{#1}
\newcommand{\FloatTok}[1]{\textcolor[rgb]{0.00,0.00,0.81}{#1}}
\newcommand{\FunctionTok}[1]{\textcolor[rgb]{0.00,0.00,0.00}{#1}}
\newcommand{\ImportTok}[1]{#1}
\newcommand{\InformationTok}[1]{\textcolor[rgb]{0.56,0.35,0.01}{\textbf{\textit{#1}}}}
\newcommand{\KeywordTok}[1]{\textcolor[rgb]{0.13,0.29,0.53}{\textbf{#1}}}
\newcommand{\NormalTok}[1]{#1}
\newcommand{\OperatorTok}[1]{\textcolor[rgb]{0.81,0.36,0.00}{\textbf{#1}}}
\newcommand{\OtherTok}[1]{\textcolor[rgb]{0.56,0.35,0.01}{#1}}
\newcommand{\PreprocessorTok}[1]{\textcolor[rgb]{0.56,0.35,0.01}{\textit{#1}}}
\newcommand{\RegionMarkerTok}[1]{#1}
\newcommand{\SpecialCharTok}[1]{\textcolor[rgb]{0.00,0.00,0.00}{#1}}
\newcommand{\SpecialStringTok}[1]{\textcolor[rgb]{0.31,0.60,0.02}{#1}}
\newcommand{\StringTok}[1]{\textcolor[rgb]{0.31,0.60,0.02}{#1}}
\newcommand{\VariableTok}[1]{\textcolor[rgb]{0.00,0.00,0.00}{#1}}
\newcommand{\VerbatimStringTok}[1]{\textcolor[rgb]{0.31,0.60,0.02}{#1}}
\newcommand{\WarningTok}[1]{\textcolor[rgb]{0.56,0.35,0.01}{\textbf{\textit{#1}}}}
% Prevent slide breaks in the middle of a paragraph:
\widowpenalties 1 10000
\raggedbottom
\AtBeginPart{
  \let\insertpartnumber\relax
  \let\partname\relax
  \frame{\partpage}
}
\AtBeginSection{
  \ifbibliography
  \else
    \begin{frame}[noframenumbering]{Contents}
    \tableofcontents[currentsection, hideothersubsections]
    \end{frame}
  \fi
}
\setlength{\emergencystretch}{3em}  % prevent overfull lines
\providecommand{\tightlist}{%
  %\setlength{\itemsep}{0pt}
  \setlength{\parskip}{0pt}
  }
\setcounter{secnumdepth}{0}

% set default figure placement to htbp
\makeatletter
\def\fps@figure{htbp}
\makeatother

\usepackage{dsfont}
\usepackage{stmaryrd}
\usepackage[normalem]{ulem}
\usepackage{fontawesome5}
\usepackage{tikz,pgfplots}
\pgfplotsset{compat=1.17}
\pgfplotsset{samples=100}
\usepackage{animate}
 \usepackage{booktabs}

\usepackage{colortbl}

\DeclareMathOperator{\Cov}{Cov}
\newcommand{\cov}[2]{\Cov\left( #1\,,\,#2 \right)}

\DeclareMathOperator{\e}{e}
\renewcommand{\P}{\mathds{P}} %Apparement \P existe déjà ?
\newcommand\N{\mathds{N}}
\newcommand\R{\mathds{R}}


\newcommand\1{\mathds{1}}
\newcommand{\E}[2][]{{\mathds{E}}_{#1}
  \def\temp{#2}\ifx\temp\empty
  \else
    \left[#2\right]
  \fi
}
\newcommand{\V}[2][]{{\mathds{V}}_{#1}
  \def\temp{#2}\ifx\temp\empty
  \else
    \left[#2\right]
  \fi
}
\newcommand\ud{\,\mathrm{d}}


% blocks
\usepackage{environ}
\usepackage[tikz]{bclogo}

\tikzstyle{titlestyle} =[draw=black!80,fill=black!20, text=black,
 right=10pt, rounded corners]
\mdfdefinestyle{symmaryboxstyle}{
	linecolor=black!80, backgroundcolor = black!5,
	skipabove=\baselineskip, innertopmargin=\baselineskip,
	innerbottommargin=\baselineskip,
	userdefinedwidth=\textwidth,
	middlelinewidth=1.2pt, roundcorner=5pt,
	skipabove={\dimexpr0.5\baselineskip+\topskip\relax},
	frametitleaboveskip=\dimexpr-\ht\strutbox\relax,
	innerlinewidth=0pt,
}
\NewEnviron{summary}{%
\begin{mdframed}[style=symmaryboxstyle]
\vspace{-0.5em}
\BODY
\end{mdframed}
}
\makeatletter
% Open `\noalign` and check for overlay specification:
\def\rowcolor{\noalign{\ifnum0=`}\fi\bmr@rowcolor}
\newcommand<>{\bmr@rowcolor}{%
    \alt#1%
        {\global\let\CT@do@color\CT@@do@color\@ifnextchar[\CT@rowa\CT@rowb}% Rest of original `\rowcolor`
        {\ifnum0=`{\fi}\@gooble@rowcolor}% End `\noalign` and gobble all arguments of `\rowcolor`.
}
% Gobble all normal arguments of `\rowcolor`:
\newcommand{\@gooble@rowcolor}[2][]{\@gooble@rowcolor@}
\newcommand{\@gooble@rowcolor@}[1][]{\@gooble@rowcolor@@}
\newcommand{\@gooble@rowcolor@@}[1][]{\ignorespaces}

\newcommand{\rowc}[1]{\only<#1>{\\\rowcolor{processblue!40}}}
%\newcommand{\rowc}[1]{{\rowcolor<#1>{processblue!30}}
\newcommand{\cellc}[1]{\only<#1>{\cellcolor{processblue!40}}}
\newcommand{\supsp}[1]{\visible<#1>{\\}}

\title{Using JDemetra+ in R: from version 2 to version 3\\
Presentation 4: SA production and quality report in R}
\ateneo{TSACE Webinar, Wednesday December 14th 2022}
\author{Anna Smyk and Tanguy Barthelemy}
\date{}


\setrellabel{}

\setcandidatelabel{}

\rel{}
\division{With the collaboration of Alain Quartier-la-tente\\}

\departement{}
\makeatletter
\let\@@magyar@captionfix\relax
\makeatother


\begin{document}
\begin{frame}[plain,noframenumbering]
\titlepage
\end{frame}

\hypertarget{introduction}{%
\section{Introduction}\label{introduction}}

\begin{frame}{Tackling Production issues}
\protect\hypertarget{tackling-production-issues}{}
Context of use :

\begin{itemize}
\item
  massive data sets
\item
  setting up a production process
\item
  annual or infra-annual reviews
\item
  producing quality reports
\item
  manual fine tuning a selected sub-set of series
\end{itemize}
\end{frame}

\hypertarget{quality-report-based-on-jdemetra-cruncher-output}{%
\section{Quality report based on JDemetra+ cruncher
output}\label{quality-report-based-on-jdemetra-cruncher-output}}

\begin{frame}{Quality report with JDCruncheR package (1/4)}
\protect\hypertarget{quality-report-with-jdcruncher-package-14}{}
\textbf{JDemetra+ Cruncher} (executable module) allows to

\begin{itemize}
\item
  update a JDemetra+ workspace (refresh policy)
\item
  export the results (series and diagnostics), without having to open
  the graphical interface and operate manually.
\end{itemize}

It can be launched in R with rjwsacruncher or JDCruncherR packages.

The \textbf{JDCruncheR package} produces a QR based on JDemetra+
cruncher output
\end{frame}

\begin{frame}[fragile]{Quality report with JDCruncheR package (2/4)}
\protect\hypertarget{quality-report-with-jdcruncher-package-24}{}
The three main functions of the JDCruncher package are:

\begin{itemize}
\item
  \texttt{extract\_QR} to extract the quality report from the csv file
  (demetra\_m.csv) that contains all JD+ diagnostics
\item
  \texttt{compute\_score} to compute a weighted score based on selected
  diagnostics and corresponding ``Good'', ``Bad'',\ldots{} modalities
\item
  \texttt{export\_xlsx} to export the quality report to Excel
\end{itemize}
\end{frame}

\begin{frame}[fragile]{Quality report with JDCruncheR (3/4): example}
\protect\hypertarget{quality-report-with-jdcruncher-34-example}{}
\footnotesize

\begin{Shaded}
\begin{Highlighting}[]
\CommentTok{\# choose the demetra\_m.csv file generated by the cruncher}
\NormalTok{QR }\OtherTok{\textless{}{-}} \FunctionTok{extract\_QR}\NormalTok{(}\StringTok{"../Output/SA"}\NormalTok{)}
\NormalTok{QR}

\NormalTok{?compute\_score }\CommentTok{\# to see how the score is calculated (formula)}
\NormalTok{QR }\OtherTok{\textless{}{-}} \FunctionTok{compute\_score}\NormalTok{(QR,}
                    \AttributeTok{n\_contrib\_score =} \DecValTok{3}\NormalTok{)}

\NormalTok{QR}

\NormalTok{QR }\OtherTok{\textless{}{-}} \FunctionTok{sort}\NormalTok{(QR, }\AttributeTok{decreasing =} \ConstantTok{TRUE}\NormalTok{, }\AttributeTok{sort\_variables =} \StringTok{"score"}\NormalTok{)}
\FunctionTok{export\_xlsx}\NormalTok{(QR,}
            \AttributeTok{file\_name =} \StringTok{"U:/quality\_report.xls"}\NormalTok{)}
\end{Highlighting}
\end{Shaded}
\end{frame}

\begin{frame}[fragile]{Quality report with JDCruncher (4/4) : example}
\protect\hypertarget{quality-report-with-jdcruncher-44-example}{}
Missing values can be ignored and conditions can be set for indicators:

\begin{Shaded}
\begin{Highlighting}[]
\CommentTok{\# oos\_mse weight reduced to 1 when the other }
\CommentTok{\# indicators are "Bad" ou "Severe"}
\NormalTok{condition1 }\OtherTok{\textless{}{-}} \FunctionTok{list}\NormalTok{(}\AttributeTok{indicator =} \StringTok{"oos\_mse"}\NormalTok{,}
                   \AttributeTok{conditions =} \FunctionTok{c}\NormalTok{(}\StringTok{"residuals\_independency"}\NormalTok{,}
                                  \StringTok{"residuals\_homoskedasticity"}\NormalTok{,}
                                  \StringTok{"residuals\_normality"}\NormalTok{),}
                   \AttributeTok{conditions\_modalities =} \FunctionTok{c}\NormalTok{(}\StringTok{"Bad"}\NormalTok{,}\StringTok{"Severe"}\NormalTok{))}
\NormalTok{BQ }\OtherTok{\textless{}{-}} \FunctionTok{compute\_score}\NormalTok{(BQ, }\AttributeTok{n\_contrib\_score =} \DecValTok{5}\NormalTok{,}
                    \AttributeTok{conditional\_indicator =} \FunctionTok{list}\NormalTok{(condition1),}
                    \AttributeTok{na.rm =} \ConstantTok{TRUE}\NormalTok{)}
\end{Highlighting}
\end{Shaded}
\end{frame}

\begin{frame}{Example of score composition}
\protect\hypertarget{example-of-score-composition}{}
\begin{table}[htbp]
  \centering
    \begin{tabular}{|m{2cm}|l|>{\centering\arraybackslash}m{2cm}|}
    \hline
    \multicolumn{2}{|c|}{Diagnostics} & Weights (out of 100) \\
    \hline
    \multirow{2}{2cm}{Pre-adjustment} & ARIMA Model Residuals & 30 \\
\cline{2-3}          & Residual Calendar Effects  & 20 \\
    \hline
    \multirow{2}{2cm}{Decomposition} & Residual seasonality & 45 \\
\cline{2-3}          & Decomposition Quality (stats M if X11) & 5 \\
    \hline
    \end{tabular}

\end{table}
\end{frame}

\begin{frame}{Customize the score computation}
\protect\hypertarget{customize-the-score-computation}{}
Practical steps if you want to customize the score computation (see
package documentation in R)

\begin{itemize}
\item
  select your indicators of interest
\item
  adjust ``good'', ``bad''\ldots threshold in JD+ GUI if necessary
\item
  by default good=0, uncertain=1, bad or severe=3
\item
  change this grading system and/or the weights directly in the package
  functions
\item
  rebuild your package
\end{itemize}

Future developments: make this functions directly customizable by the
user

In this QR only diagnostics are taken into account, revisions and
numerical effects of potential parameter tuning still have to be
analysed
\end{frame}

\hypertarget{sa-production-in-r}{%
\section{SA production in R}\label{sa-production-in-r}}

\begin{frame}{SA production (fully?) in R}
\protect\hypertarget{sa-production-fully-in-r}{}
A request which comes back all the time

\begin{itemize}
\tightlist
\item
  better (?) automation when the remainder of the production process
  (outside of SA) is also done in R
\end{itemize}

We will contrast

\begin{itemize}
\item
  ``old fashion set-up'': workspace created in GUI, readable with GUI
  and refreshable with the cruncher (functions from R packages might be
  used as auxiliary tools, e.g ``instant read''..)
\item
  ``full R set-up'': no workspace structure, time series objects only
\end{itemize}
\end{frame}

\begin{frame}{Data format and portability}
\protect\hypertarget{data-format-and-portability}{}
Workspace created in GUI:

\begin{itemize}
\tightlist
\item
  rigid data structure (series order constraints)
\item
  physical path to data (not portable)
\end{itemize}

Time-series objects in R: complete flexibility and portability (Sharing
R projects can be done easily)
\end{frame}

\begin{frame}{Fine-tuning specifications}
\protect\hypertarget{fine-tuning-specifications}{}
Context of use :

\begin{itemize}
\item
  SA processing first set-up or annual review
\item
  massive data set
\item
  each series has specific (pre-determined) parameters: pre-specified
  outliers, calendar regressors
\end{itemize}

Fine-tuning specifications:

\begin{itemize}
\item
  In a classical Workspace: not easy, need for an auxiliary tool (in
  java for example)
\item
  In a full R set-up: easier to write code generating specific
  (different) specifications in large number and link them to series
\end{itemize}
\end{frame}

\begin{frame}{Estimation and Refreshing data}
\protect\hypertarget{estimation-and-refreshing-data}{}
In a classical workspace :

\begin{itemize}
\tightlist
\item
  very easy and fast with the Cruncher
\end{itemize}

In a full R set up:

\begin{itemize}
\item
  more code needed, as fast ? (probably)
\item
  refresh policies (V3) even more flexible
\item
  output is directly available in R for further processing
\end{itemize}
\end{frame}

\begin{frame}{Annual review, selective editing and manual fine tuning}
\protect\hypertarget{annual-review-selective-editing-and-manual-fine-tuning}{}
Step 1: comparing old and new sets of parameters (classical
``current''vs ``automatic'' reestimation)

\begin{itemize}
\item
  easy in both contexts using R (reading W with RJDemetra) :
\item
  score computation with diagnostics : JDCruncher or adapted version to
  R objects
\end{itemize}

Step 2: select important series and fine-tune manually

\begin{itemize}
\item
  classical Workspace GUI for manual expertise: significant asset
\item
  R setup: visual feedback not as rich, nor multi-layer, nor as easy to
  navigate as GUI
\end{itemize}
\end{frame}

\hypertarget{conclusion}{%
\section{Conclusion}\label{conclusion}}

\begin{frame}{On production in R}
\protect\hypertarget{on-production-in-r}{}
Main asset of WS-GUI-Cruncher set up:

\begin{itemize}
\item
  GUI rich and multi-level feedback for manual fine tuning
\item
  use data structure rigidity at your advantage (easier to keep track of
  production with workspaces ?)
\end{itemize}

Assets of ``Full R set up''

\begin{itemize}
\item
  portability
\item
  direct availability of objects in R
\end{itemize}
\end{frame}

\hypertarget{overall-conclusion}{%
\subsection{Overall conclusion}\label{overall-conclusion}}

\begin{frame}[fragile]{Upgrading from v2 to v3}
\protect\hypertarget{upgrading-from-v2-to-v3}{}
What is new ?

\begin{itemize}
\item
  Tools : tests, arima estimation
\item
  More specific and \texttt{fast()} functions
\item
  Refresh policies, even more flexible
\end{itemize}

Upgrading code from v2 to v3

Cost of code conversion

\begin{itemize}
\item
  global functions very similar (arguments)
\item
  customization process significantly different
\item
  organisation of stored objects significantly different
\end{itemize}
\end{frame}

\begin{frame}{Future developments}
\protect\hypertarget{future-developments}{}
what is missing ?

\begin{itemize}
\item
  functions for handling workspaces in v3
\item
  update/reframe auxiliary packages (rdjworkspace, ex ggdemetra3)
\end{itemize}

\ldots open discussion

Possible Contributions

\begin{itemize}
\item
  Testing it and reporting issues (rest\_rjd3 repo on GitHub)
\item
  Developing new tools (other packages, new functions, etc.)
\end{itemize}
\end{frame}

\begin{frame}{Resources and documentation}
\protect\hypertarget{resources-and-documentation}{}
\textbf{Webinar Resources} on GitHub:

\begin{itemize}
\item
  slides with code (rmd files)
\item
  additional references: Beamers, Working Paper on v2 set of tools (jan
  2021) will be updated to v3 (March 2023)
\end{itemize}

\url{https://github.com/annasmyk/Tsace_RJD_Webinar_Dec22}

Coming soon: \textbf{JDemetra+ NEW online documentation} ``first
release'' on Thursday December 22nd:

\url{https://jdemetra-new-documentation.netlify.app/}

Restricted scope : Chapters on SA (incl HF) and Chapters on Tools (GUI,
R packages and plug-ins) It will grow from there\ldots{}
\end{frame}

\begin{frame}{BLOG JDemetra+ universe}
\protect\hypertarget{blog-jdemetra-universe}{}
We are starting a Blog: \textbf{JDemetra+ universe}, the missing piece
between Cros Portal and GitHub pages\\
\url{https://jdemetra-universe-blog.netlify.app/}

\begin{itemize}
\item
  can be used for problem/solution/insights sharing (comments available
  if logged into GitHub)
\item
  guest posts welcome
\item
  we will link ``all'' presentations about JDemetra+ in conferences /
  workshops * (If you give a talk about JD+ let us know\ldots)
\end{itemize}
\end{frame}

\begin{frame}[noframenumbering]{Thank you for your attention}
\protect\hypertarget{thank-you-for-your-attention}{}
Packages \faIcon{r-project}{}:

\begin{columns}[T]
\begin{column}{0.4\textwidth}
\href{https://github.com/palatej/rjd3toolkit}{\faGithub{} palatej/rjd3toolkit}

\href{https://github.com/palatej/rjd3modelling}{\faGithub{} palatej/rjd3modelling}

\href{https://github.com/palatej/rjd3sa}{\faGithub{} palatej/rjd3sa}

\href{https://github.com/palatej/rjd3arima}{\faGithub{} palatej/rjd3arima}

\href{https://github.com/palatej/rjd3x13}{\faGithub{} palatej/rjd3x13}

\href{https://github.com/palatej/rjd3tramoseats}{\faGithub{} palatej/rjd3tramoseats}

\href{https://github.com/palatej/rjdemetra3}{\faGithub{} palatej/rjdemetra3}
\end{column}

\begin{column}{0.5\textwidth}
\href{https://github.com/palatej/rjdfilters}{\faGithub{} palatej/rjdfilters}

\href{https://github.com/palatej/rjd3sts}{\faGithub{} palatej/rjd3sts}

\href{https://github.com/palatej/rjd3stl}{\faGithub{} palatej/rjd3stl}

\href{https://github.com/palatej/rjd3highfreq}{\faGithub{} palatej/rjd3highfreq}

\href{https://github.com/palatej/rjd3bench}{\faGithub{} palatej/rjd3bench}

\href{https://github.com/AQLT/ggdemetra3}{\faGithub{} AQLT/ggdemetra3}
\end{column}
\end{columns}
\end{frame}

\end{document}
