\PassOptionsToPackage{unicode=true}{hyperref} % options for packages loaded elsewhere
\PassOptionsToPackage{hyphens}{url}
%
\documentclass[10pt,xcolor=table,color={dvipsnames,usenames},ignorenonframetext,usepdftitle=false,french]{beamer}
\setbeamertemplate{caption}[numbered]
\setbeamertemplate{caption label separator}{: }
\setbeamercolor{caption name}{fg=normal text.fg}
\beamertemplatenavigationsymbolsempty
\usepackage{caption}
\captionsetup{skip=0pt,belowskip=0pt}
%\setlength\abovecaptionskip{-15pt}
\usepackage{lmodern}
\usepackage{amssymb,amsmath,mathtools,multirow}
\usepackage{float,hhline}
\usepackage{tikz}
\usepackage{mathtools}
\usepackage{ifxetex,ifluatex}
\usepackage{fixltx2e} % provides \textsubscript
\ifnum 0\ifxetex 1\fi\ifluatex 1\fi=0 % if pdftex
  \usepackage[T1]{fontenc}
  \usepackage[utf8]{inputenc}
  \usepackage{textcomp} % provides euro and other symbols
\else % if luatex or xelatex
  \usepackage{unicode-math}
  \defaultfontfeatures{Ligatures=TeX,Scale=MatchLowercase}
\fi
\usetheme[coding=utf8,language=english,
,titlepagelogo=img/SACElogo
]{TorinoTh}
% use upquote if available, for straight quotes in verbatim environments
\IfFileExists{upquote.sty}{\usepackage{upquote}}{}
% use microtype if available
\IfFileExists{microtype.sty}{%
\usepackage[]{microtype}
\UseMicrotypeSet[protrusion]{basicmath} % disable protrusion for tt fonts
}{}
\IfFileExists{parskip.sty}{%
\usepackage{parskip}
}{% else
\setlength{\parindent}{0pt}
\setlength{\parskip}{6pt plus 2pt minus 1pt}
}
\usepackage{hyperref}
\hypersetup{
            pdfauthor={Anna Smyk and Tanguy Barthelemy},
            pdfborder={0 0 0},
            breaklinks=true}
\urlstyle{same}  % don't use monospace font for urls
\newif\ifbibliography
\newlength{\cslhangindent}
\setlength{\cslhangindent}{1.5em}
\newlength{\csllabelwidth}
\setlength{\csllabelwidth}{3em}
\newenvironment{CSLReferences}[2] % #1 hanging-ident, #2 entry spacing
 {% don't indent paragraphs
  \setlength{\parindent}{0pt}
  % turn on hanging indent if param 1 is 1
  \ifodd #1 \everypar{\setlength{\hangindent}{\cslhangindent}}\ignorespaces\fi
  % set entry spacing
  \ifnum #2 > 0
  \setlength{\parskip}{#2\baselineskip}
  \fi
 }%
 {}
% Prevent slide breaks in the middle of a paragraph:
\widowpenalties 1 10000
\raggedbottom
\AtBeginPart{
  \let\insertpartnumber\relax
  \let\partname\relax
  \frame{\partpage}
}
\AtBeginSection{
  \ifbibliography
  \else
    \begin{frame}[noframenumbering]{Contents}
    \tableofcontents[currentsection, hideothersubsections]
    \end{frame}
  \fi
}
\setlength{\emergencystretch}{3em}  % prevent overfull lines
\providecommand{\tightlist}{%
  %\setlength{\itemsep}{0pt}
  \setlength{\parskip}{0pt}
  }
\setcounter{secnumdepth}{0}

% set default figure placement to htbp
\makeatletter
\def\fps@figure{htbp}
\makeatother

\usepackage{dsfont}
\usepackage{stmaryrd}
\usepackage[normalem]{ulem}
\usepackage{fontawesome5}
\usepackage{tikz,pgfplots}
\pgfplotsset{compat=1.17}
\pgfplotsset{samples=100}
\usepackage{animate}
 \usepackage{booktabs}

\usepackage{colortbl}

\DeclareMathOperator{\Cov}{Cov}
\newcommand{\cov}[2]{\Cov\left( #1\,,\,#2 \right)}

\DeclareMathOperator{\e}{e}
\renewcommand{\P}{\mathds{P}} %Apparement \P existe déjà ?
\newcommand\N{\mathds{N}}
\newcommand\R{\mathds{R}}


\newcommand\1{\mathds{1}}
\newcommand{\E}[2][]{{\mathds{E}}_{#1}
  \def\temp{#2}\ifx\temp\empty
  \else
    \left[#2\right]
  \fi
}
\newcommand{\V}[2][]{{\mathds{V}}_{#1}
  \def\temp{#2}\ifx\temp\empty
  \else
    \left[#2\right]
  \fi
}
\newcommand\ud{\,\mathrm{d}}


% blocks
\usepackage{environ}
\usepackage[tikz]{bclogo}

\tikzstyle{titlestyle} =[draw=black!80,fill=black!20, text=black,
 right=10pt, rounded corners]
\mdfdefinestyle{symmaryboxstyle}{
	linecolor=black!80, backgroundcolor = black!5,
	skipabove=\baselineskip, innertopmargin=\baselineskip,
	innerbottommargin=\baselineskip,
	userdefinedwidth=\textwidth,
	middlelinewidth=1.2pt, roundcorner=5pt,
	skipabove={\dimexpr0.5\baselineskip+\topskip\relax},
	frametitleaboveskip=\dimexpr-\ht\strutbox\relax,
	innerlinewidth=0pt,
}
\NewEnviron{summary}{%
\begin{mdframed}[style=symmaryboxstyle]
\vspace{-0.5em}
\BODY
\end{mdframed}
}
\makeatletter
% Open `\noalign` and check for overlay specification:
\def\rowcolor{\noalign{\ifnum0=`}\fi\bmr@rowcolor}
\newcommand<>{\bmr@rowcolor}{%
    \alt#1%
        {\global\let\CT@do@color\CT@@do@color\@ifnextchar[\CT@rowa\CT@rowb}% Rest of original `\rowcolor`
        {\ifnum0=`{\fi}\@gooble@rowcolor}% End `\noalign` and gobble all arguments of `\rowcolor`.
}
% Gobble all normal arguments of `\rowcolor`:
\newcommand{\@gooble@rowcolor}[2][]{\@gooble@rowcolor@}
\newcommand{\@gooble@rowcolor@}[1][]{\@gooble@rowcolor@@}
\newcommand{\@gooble@rowcolor@@}[1][]{\ignorespaces}

\newcommand{\rowc}[1]{\only<#1>{\\\rowcolor{processblue!40}}}
%\newcommand{\rowc}[1]{{\rowcolor<#1>{processblue!30}}
\newcommand{\cellc}[1]{\only<#1>{\cellcolor{processblue!40}}}
\newcommand{\supsp}[1]{\visible<#1>{\\}}

\title{Using JDemetra+ in R: from version 2 to version 3\\
Presentation 2: Seasonal adjustment in R}
\ateneo{TSACE Webinar, Wednesday December 14th 2022}
\author{Anna Smyk and Tanguy Barthelemy}
\date{}


\setrellabel{}

\setcandidatelabel{}

\rel{}
\division{With the collaboration of Alain Quartier-la-tente\\}

\departement{}
\makeatletter
\let\@@magyar@captionfix\relax
\makeatother


\begin{document}
\begin{frame}[plain,noframenumbering]
\titlepage
\end{frame}

\hypertarget{introduction}{%
\section{Introduction}\label{introduction}}

\begin{frame}{Outline table}
\protect\hypertarget{outline-table}{}
\end{frame}

\begin{frame}{Data formats}
\protect\hypertarget{data-formats}{}
here, no workspace structure - assets - shortcomings
\end{frame}

\begin{frame}{SA process}
\protect\hypertarget{sa-process}{}
\begin{itemize}
\tightlist
\item
  identifying seasonality
\item
  pre treatement
\item
  decomposition
\item
  quality assessment
\end{itemize}

comp with GUI main panels ?
\end{frame}

\begin{frame}{rjd3 suite of packages for SA}
\protect\hypertarget{rjd3-suite-of-packages-for-sa}{}
in v2 :

in v3: more tools (tests,\ldots)
\end{frame}

\hypertarget{sa-or-time-series-tools}{%
\section{SA (or Time series) tools}\label{sa-or-time-series-tools}}

\begin{frame}{Identifying seasonal patterns}
\protect\hypertarget{identifying-seasonal-patterns}{}
\end{frame}

\begin{frame}{Normality test}
\protect\hypertarget{normality-test}{}
\end{frame}

\begin{frame}{Autocorrelation}
\protect\hypertarget{autocorrelation}{}
\end{frame}

\hypertarget{x13}{%
\section{X13}\label{x13}}

\hypertarget{quick-launch-with-default-specifications}{%
\subsection{Quick Launch with default
specifications}\label{quick-launch-with-default-specifications}}

\begin{frame}{Quick Launch with default specifications}
\protect\hypertarget{quick-launch-with-default-specifications-1}{}
\begin{itemize}
\tightlist
\item
  x13
\item
  regarima
\item
  x11 (one less spec in default x13)
\end{itemize}
\end{frame}

\hypertarget{rerieving-output-and-data-visualization}{%
\subsection{Rerieving output and data
visualization}\label{rerieving-output-and-data-visualization}}

\begin{frame}{Output structure v2}
\protect\hypertarget{output-structure-v2}{}
show the list of lists do a new version
\end{frame}

\begin{frame}{Output structure v3 (cf txt file)}
\protect\hypertarget{output-structure-v3-cf-txt-file}{}
show the NEW list of lists

highlight deifferences: - specs - specs direct accessible + 2 concepts
(spec in v12 was point spec;, more about this in refresh section)
\end{frame}

\begin{frame}{Retrieve output series}
\protect\hypertarget{retrieve-output-series}{}
\begin{itemize}
\tightlist
\item
  final intermediate computations
\item
  from preadjustement
\end{itemize}

highlight differences v2 vs v3
\end{frame}

\begin{frame}{Retrieve Diagnostics}
\protect\hypertarget{retrieve-diagnostics}{}
\end{frame}

\begin{frame}{Plots and data visualisation}
\protect\hypertarget{plots-and-data-visualisation}{}
in v2 in v3 : .mostly in ggdemetra3 for now ..
\end{frame}

\hypertarget{customizing-specifications}{%
\subsection{Customizing
specifications}\label{customizing-specifications}}

\begin{frame}{Customizing specifications}
\protect\hypertarget{customizing-specifications-1}{}
v2: - step 1: extract spec - step 2: use the spec function with
user-defined arguments v3: - use direct set\_ functions

check what can be set - a - b - ouliers
\end{frame}

\begin{frame}{Adding a context}
\protect\hypertarget{adding-a-context}{}
new in v3, relevant ?
\end{frame}

\begin{frame}{Customizing calendar regressors}
\protect\hypertarget{customizing-calendar-regressors}{}
in v2 in v3
\end{frame}

\begin{frame}{Intervention variables}
\protect\hypertarget{intervention-variables}{}
in v2

in v3 : still a bug
\end{frame}

\begin{frame}{User-defined parameters: summary}
\protect\hypertarget{user-defined-parameters-summary}{}
BILAN - what's new ? - whats's missing ?
\end{frame}

\hypertarget{refreshing-data}{%
\subsection{Refreshing data}\label{refreshing-data}}

\begin{frame}{Refreshing data}
\protect\hypertarget{refreshing-data-1}{}
new feature of v3

\begin{itemize}
\tightlist
\item
  new handling of spec (no extraction needed)
\item
  notion of point spec and domain spec
\item
  in v2 could only retrieve point spec
\item
  generatingn new spec for refesh
\item
  new estimation
\end{itemize}
\end{frame}

\hypertarget{tramo-seats}{%
\section{Tramo-seats}\label{tramo-seats}}

\begin{frame}{rjd3tramoseats package}
\protect\hypertarget{rjd3tramoseats-package}{}
here (optionnal) what is different from the way rjd3x13 operates
\end{frame}

\hypertarget{sa-of-high-frequency-data}{%
\section{SA of High-Frequency data}\label{sa-of-high-frequency-data}}

\begin{frame}{SA of High-Frequency data}
\protect\hypertarget{sa-of-high-frequency-data-1}{}
tool oriented
\end{frame}

\hypertarget{generating-user-defined-auxilary-variables}{%
\section{Generating User-defined auxilary
variables}\label{generating-user-defined-auxilary-variables}}

\hypertarget{calendars}{%
\subsection{calendars}\label{calendars}}

\begin{frame}{calendars}
\protect\hypertarget{calendars-1}{}
here new functionnality of v3, rjd3modelling pacakage
\end{frame}

\hypertarget{outliers-and-intervention-variables}{%
\subsection{outliers and intervention
variables}\label{outliers-and-intervention-variables}}

\begin{frame}{outliers and intervention variables}
\protect\hypertarget{outliers-and-intervention-variables-1}{}
(using this variables already presented, now focus on generation)

intervention bug in rj3modelling ?
\end{frame}

\hypertarget{conclusion}{%
\section{Conclusion}\label{conclusion}}

\begin{frame}{Conclusion on SA in R}
\protect\hypertarget{conclusion-on-sa-in-r}{}
What has v3 brought to the table ?
\end{frame}

\end{document}
